%
%	Praxisbezug
%

\pagebreak
\section{Data Analysis}

\onehalfspacing

\subsection{Blog Post Data Overview}

Before enriching the data for analysis, let's have a look at the raw web analytics data and KIPs for the various posts:

\begin{figure}[H]
\centering
\caption {Raw Blog Post Data}
\includegraphics[width=\linewidth]{images/analysis-raw.png}
\label{fig:analysisRaw}
\end{figure}

I have excluded all top-level access, such as from "/wordpress/" - visits by accessing the blog itself and then scrolling down can not be attributed to a certain post and thus not be enriched with category or sentiment data.

\subsection{Data Wrangling}

Adding category information from the blog to the raw data was relatively easy and a straight-forward exercise with vi.

Blog posts on a WordPress blog, however, cannot easily be extracted. Wordpress offers an export function, but that will result in an archive in XML format, which is not suitable for text analysis.

After several tests, I found that the best way to extract the text from the actual posts was to use a text-based browser; I chose elinks for the task.

We already have the post URL in the data above, so extracting was done with a simple shell script:

\verb|elinks https://.../2020/08/16/travel/ -dump > travel|

\verb|...|

As a the next step I created a small Python script to analyze the text and return the compoung sentiment value:

\begin{lstlisting}[caption=Vader Sentiment Compound, frame=single, basicstyle=\ttfamily]
from vaderSentiment.vaderSentiment 
  import SentimentIntensityAnalyzer

analyzer = SentimentIntensityAnalyzer()

filenames = [ "anti-sexist-social-club",
              "biden-harris",
              ...
              "xfce" ]

for filename in filenames:
  fd = open(filename)
  blogPost = fd.read()
  sentiment = analyzer.polarity_scores(blogPost)
  print(filename + " " + 
    str(int(sentiment['compound'] * 10)))

\end{lstlisting}

This script will return the compound sentiment value for the text from the VADER analysis, one row per file.

\verb|anti-sexist-social-club 9|

\verb|biden-harris 6|

\verb|...|

\verb|xfce 9|

Adding the sentiment to the raw data again was a task for vi and sed.

As a result, we now have a data set of all blog posts, the associated KPI values, their categories and their sentiment.

\subsection{Blog Post Category Correlation}

Now that we've enriched the blog post data with category and sentiment information let's look at our KPIs by category. The encoding was as follows:

\begin{itemize}
\item Climate and politics: 1
\item Cloud: 2
\item Social Media and culture: 3
\item Covid-19: 4
\end{itemize}

\begin{figure}[H]
\centering
\caption {Category Analysis}
\includegraphics[width=\linewidth]{images/analysis-category.png}
\label{fig:analysisCategory}
\end{figure}

The top-performing category with the most unique visitors, the most page views, and the lowest bounce rate is Climate and Politics; the second-best performing category in all KPIs is Covid-19. The two other categories fall way behind - my posts regarding either Cloud or Social Media don't seem to be interesting to my readers.

\subsection{Blog Post Sentiment Correlation}

For the two best performing categories, Climate and Politics, and Covid-19, let's have a look at the KPIs broken down by sentiment, Climate first.

The sentiment is a compound value from the NLP analysis and ranges between -10 and 10, from total negative to total positive.

\begin{figure}[H]
\centering
\caption {Sentiment Analysis - Climate}
\includegraphics[width=\linewidth]{images/analysis-sentiment-climate.png}
\label{fig:sentimentClimate}
\end{figure}

If we compare the number of Unique Visitors for a Climate post with positive and negative sentiment, we cannot observe a significantly large difference (62 vs. 74). When we look at the number of Page Views (98 vs. 145), we can see a difference, with favor towards climate action posts with a more negative sentiment. There seems to be no influence from the posts' sentiment on the average Bounce Rate.

Now let's do the same analysis for posts regarding Covid-19 and the current pandemic.

\begin{figure}[H]
\centering
\caption {Sentiment Analysis - Covid-19}
\includegraphics[width=\linewidth]{images/analysis-sentiment-covid.png}
\label{fig:sentimentCovid}
\end{figure}

Now there's a visible difference - both Unique Visitors (45 vs. 21) and Page Views (79 vs. 28) are significantly higher for blog posts on Covid-19 with a more positive sentiment. Again, there seems to be no influence from the posts' sentiment on the average Bounce Rate.

\subsection{Results}

Text.

\subsection{Predictions}

Text.

\subsection{Data Source}

Again you can find the raw data and all charts from this chapter in the data notebook here: \href{https://plausible.io/chfrank.net}{WTA Chapter 4}
