%
%	Einfuehrung
%

\pagebreak
\section{Web Traffic Analysis}

\onehalfspacing

\subsection{Web Analytics}

In this paper, I will look at my blog's traffic data, analyze that data, and then lay the foundation to predict a blog post's performance based on its content.

Web analytics is the domain of the big search engines, and Google Analytics and Google AdSense are the market leaders, followed by Microsoft Bing Web Analytics. In all of Social Media and Social Media Marketing, web analytics plays a crucial role in evaluating a website's performance and forms the basis of automated advertising placement.

Advertising, as much as we might dislike it, pays for most of the content we consume.

Page views, bounce rate, and unique visitors are key metrics to evaluate a website and the currency that fuels the internet. Every marketeer or web site owner will use these metrics to analyze performance and identify areas for growth; many tools for analysis have become available in the last couple of years, some of them open-source, some closed-source.

Generally speaking, more traffic can potentially lead to more business opportunities. It is mandatory for a commercial website to monitor its web analytics data daily and act immediately on any anomaly.

However, a blog does not necessarily have commercial interests and might be an outlet for personal interests or interactions. Why would we want to look at web analytics anyway?

\subsection{Social Media and Loneliness}

In the current COVID-19 pandemic, social distancing is a key element in containing the virus's spread. Social distancing over a long period of time can increase loneliness and significantly affect people's health negatively, according to a recent study conducted by the American Psychological Association.\footnote{See \textit{Luchetti, M. (2020)}: The trajectory of loneliness in response to COVID-19. \cite{apaLoneliness}}

In the study, the researchers formulate the hypothesis that an increase in perceived support from others can offset loneliness during the required isolation.

One element to offset the effects of loneliness is increased interaction on social media and virtual meetings with video. Social media interaction includes reading blogs - the more engaging a blog post is, the more chances it has to reach people to whom it will be entertaining or otherwise beneficial.

Thus this analysis aims to get an answer for my blog on the question "On which subject(s) should I post to increase my reach?". We will be using visualization and correlation as the primary means of analysis to increase the value of the blog for others.

\subsection{Gender-neutral Pronouns}

As we move towards a more inclusive and gender-fluid society, it's time to rethink the usage of gendered pronouns in scientific texts. Two well-known professors from UCLA, Abigail C. Saguy and Juliet A. Williams, argue that it makes a lot of sense to use singular they/them instead: "The universal singular they is inclusive of people who identify as male, female or nonbinary."\footnote{\textit{Saguy, A. (2020)}: Why We Should All Use They/Them Pronouns. \cite{pronouns}} Throughout this paper, I'll attempt to follow their suggestion and invite my readers to do the same in future articles, and support an inclusive approach through gender-neutral language. Thank you!

\subsection{Cultural Bias}

As we start to rely on data more and more to make business decisions, train our machine learning models and make predictions, we need to pay more attention to the cultural bias in the data that we use.\footnote{See \textit{Johnson, K. (2020)}: AI research survey finds machine learning needs a culture change. \cite{aiResearch}} The study quoted in this article makes a case for data sets that respect the context they were created in and their creators' privacy rights. \footnote{See \textit{Paullada, A. (2020)}: Data and its (dis)contents: A survey of data-set development and use in machine learning research. \cite{cornellStudy}} As a side note, it also points out that data curated with more respect will also most likely be more expensive.

The content in my blog and the people who read it are influenced by the fact that I am White and male. The bias affects both source (content) and destination (analysis) and thus does not distort the results.

Other than that is the data in use in this paper utterly free from cultural bias. 

I have collected it for myself, and it is based on my content. The data also fully respects the blog readers' privacy as I do not collect any personally-identifiable data points; I assume that it would satisfy the study's quest for purposefully curated data.

However, the data is not free from personal bias - as you follow along, please feel free to interpret the data in different ways and draw your own conclusions; I will include links to all the data throughout the text.
