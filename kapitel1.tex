%
%	Einfuehrung
%

\pagebreak
\section{Introduction into Web Traffic Analysis}

\onehalfspacing

\subsection{Pronouns}

As we move towards a more gender-fluid society, it's time to rethink the usage of gendered pronouns in scientific texts. Two well-known professors from UCLA, Abigail C. Saguy, and Juliet A. Williams argue that it makes a lot of sense to use singular they/them instead: "The universal singular they is inclusive of people who identify as male, female or nonbinary."\footnote{\textit{Saguy, A. (2020)}: Why We Should All Use They/Them Pronouns. \cite{pronouns}} Throughout this text, I'll attempt to follow that suggestion and invite my readers to do the same for their papers, and support gender inclusivity through gender-neutral language. A strong focus on diversity and inclusion will significantly benefit your IT organization and help you find and retain talent. Thank you!

\subsection{Web Traffic Analysis}

Text

\subsection{Reseaech Question}

Text
