%
%	Einfuehrung
%

\pagebreak
\section{Web Traffic Analysis}

\onehalfspacing

\subsection{Web Analytics}

In this paper, I will look at the traffic data of my blog, analyze that data, and then lay the foundation to predict a blog post's performance based on its content.

Web analytics is the domain of the big search engines, and Google Analytics and Google AdSense are the market leaders. In all of Social Media and Social Media Marketing, web analytics plays a key role in evaluating a web site's performance and forms the basis of automated advertising placement.

Page views, bounce rate and unique visitors are key metrics to evaluate a web site and the currency that fuels the internet. Every marketeer or web site owner will use these metrics to analyze performance and identify areas for growth; a lot of tools for analysis have become available in the last couple of years, some of them open-source, some closed-source.

Generally speaking, more traffic can potentially lead to more business opportunities. For a commercial web site, it is absolutely mandatory to monitor its web analytics data on a daily basis and act immediately on any anomaly.

However, a blog does not necessarily have commercial interests and might just be an outlet for personal interests or interactions. Why would we want to look at web analytics anyway?

\subsection{Social Media and Loneliness}

In the current COVID-19 pandemic, social distancing is a key element in containing the spread of the virus. Social distancing over a long period of time can increase loneliness and significantly affect people's health negatively, according to a recent study conducted by the American Psychological Association.\footnote{See \textit{Luchetti, M. (2020)}: The trajectory of loneliness in response to COVID-19. \cite{apaLoneliness}}

In the study, the researchers formulate the hypothesizes, that an increase in perceived support from others can offset the feeling of loneliness during the required isolation.

One element to offset the effects of loneliness is increased interaction on social media, in addition to virtual video meetings. Social media interaction includes blogs. The more engaging a blog post is, the more chances are that it can reach people to whom it will be entertaining or otherwise beneficial.

Thus the aim of this analysis is to get an answer for my blog on the question "On which subject should I post to increase my reach?", using visualization and correlation as the primary means, in the hope of increasing its value for others.

\subsection{Gender-neutral Pronouns}

As we move towards a more inclusive and gender-fluid society, it's time to rethink the usage of gendered pronouns in scientific texts. Two well-known professors from UCLA, Abigail C. Saguy, and Juliet A. Williams argue that it makes a lot of sense to use singular they/them instead: "The universal singular they is inclusive of people who identify as male, female or nonbinary."\footnote{\textit{Saguy, A. (2020)}: Why We Should All Use They/Them Pronouns. \cite{pronouns}} Throughout this paper, I'll attempt to follow their suggestion and am inviting my readers to do the same in future papers, and support an inclusive approach through gender-neutral language. Thank you!
