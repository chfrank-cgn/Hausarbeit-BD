%
%	Fazit
%

\pagebreak
\section{Summary}

\onehalfspacing

From the analysis we can draw the conclusion that it does make sense to focus the posts on my blog on the issues of the two main crisis of our time, the Climate Emergency and the COVID-19 pandemic.

The result is not entirely surprising, as these are the issues that are mostly talked about on all media and are on everybody's mind all the time. However, having the assumptions being proved by data science through a thorough data analysis helps a lot and will guide me in further postings and the development of the blog.

The power is in the data - even for a small blog (and thus a small data set), analyzing the data is a worthwhile thing to do and will lead to interesting and actionable results.

As we saw, Plausible is a very valuable open-source tool for web analytics. There are many other open-source tools available to support data science and analysis; there is also a big community around these tools and the subject of Big Data and Data Science. 

In this paper I was merely able to scratch the surface but I do hope I was able to provide you with at least some valuable insights and pointers to start with; all available raw data is in the two data notebooks and in the side bar of my blog.

Happy Analysis!
